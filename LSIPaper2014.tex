\documentclass[preprint2]{aastex}

%% preprint2 produces a double-column, single-spaced document:

%% \documentclass[preprint2]{aastex}

%% Sometimes a paper's abstract is too long to fit on the
%% title page in preprint2 mode. When that is the case,
%% use the longabstract style option.

%% \documentclass[preprint2,longabstract]{aastex}

%% If you want to create your own macros, you can do so
%% using \newcommand. Your macros should appear before
%% the \begin{document} command.
%%
%% If you are submitting to a journal that translates manuscripts
%% into SGML, you need to follow certain guidelines when preparing
%% your macros. See the AASTeX v5.x Author Guide
%% for information.

\usepackage{graphicx}
%\usepackage{caption}
%\usepackage{subcaption}
\usepackage{subfigure}
\bibliographystyle{apj}
\usepackage{natbib}

\newcommand{\vdag}{(v)^\dagger}
\newcommand{\myemail}{}
\newcommand{\pflux}{~photons cm$^{-2}$ s$^{-1}$}
\newcommand{\lsi}{LS~I~+61$^{\circ}$~303}
\newcommand{\gev}{\,GeV}
\newcommand{\tev}{\,TeV}

%% You can insert a short comment on the title page using the command below.


%% If you wish, you may supply running head information, although
%% this information may be modified by the editorial offices.
%% The left head contains a list of authors,
%% usually a maximum of three (otherwise use et al.).  The right
%% head is a modified title of up to roughly 44 characters.
%% Running heads will not print in the manuscript style.

\shorttitle{Exceptionally Bright TeV Emission From the Binary LS I +61 303}
\shortauthors{}

%% This is the end of the preamble.  Indicate the beginning of the
%% paper itself with \begin{document}.

\begin{document}

%% LaTeX will automatically break titles if they run longer than
%% one line. However, you may use \\ to force a line break if
%% you desire.

\title{Exceptionally Bright TeV Emission From the Binary \lsi{}}

%% Use \author, \affil, and the \and command to format
%% author and affiliation information.
%% Note that \email has replaced the old \authoremail command
%% from AASTeX v4.0. You can use \email to mark an email address
%% anywhere in the paper, not just in the front matter.
%% As in the title, use \\ to force line breaks.

\author{
Andy Smith\altaffilmark{1},
Anna OFdB\altaffilmark{2},
VERITAS Collaboration\altaffilmark{3}
}

%% Notice that each of these authors has alternate affiliations, which
%% are identified by the \altaffilmark after each name.  Specify alternate
%% affiliation information with \altaffiltext, with one command per each
%% affiliation.

\altaffiltext{1}{America}
\altaffiltext{2}{Germany}
\altaffiltext{3}{Everywhere}

%% Mark off your abstract in the ``abstract'' environment. In the manuscript
%% style, abstract will output a Received/Accepted line after the
%% title and affiliation information. No date will appear since the author
%% does not have this information. The dates will be filled in by the
%% editorial office after submission.

\begin{abstract}
The TeV binary system \lsi{} is known for its regular, although not entirely understood, non-thermal emission pattern, which traces the orbital period of the compact object in its 26.5 day orbit around its Be star companion. When active in the TeV regime, the system typically presents elevated emission around apastron passage with flux levels in the 5\,--\,15\,\% Crab Nebula range (above 350 GeV). In this article, VERITAS observations of \lsi{} taken in late 2014 are presented, in which bright TeV flares around apastron at flux levels peaking above $30\%$ of the Crab Nebula flux were detected. This is the brightest such activity ever seen in the TeV regime. Low-level TeV emission is also detected close to periastron. The strong outbursts have rise times of 1\,--\,2 days and a typical 1\,--\,2 day duration; during the 2014 observations \lsi{} was seen to go from a quiescent TeV state to one in which $>10$\tev{} emission was detected from the system. This short acceleration time for particles (for galactic scale objects) provides strong constraints on the nature of the accelerating mechanism in \lsi{}.
\end{abstract}


\keywords{}

\section{Introduction}

The current generation of Imaging Atmospheric Cherenkov Telescopes (IACTs) has opened up the study of high mass X-ray binary star systems which also present TeV emission on various timescales. The class of TeV binaries is quite sparse, consisting of only a handful of sources: LS 5039, PSR B1259-63, \lsi{}, HESS J0632+057 and HESS J1018-589. Of these, only the compact object of PSR B1259-63 has been firmly identified as a pulsar; there is still a large degree of ambiguity concerning the nature of the compact object within the other systems, and consequently, the fundamental setup which produces the TeV emission along with its characteristic variability on the orbital period timescale. For instance, the presence of a pulsar within a given TeV binary indicates that the emission in the system is generated by the shock formed at the interface between the pulsar and stellar winds. The orbital variability is therefore driven principally by the varying density of the stellar wind that the pulsar encounters in its orbit. In the case of a black hole companion, the emission is driven by an accretion-powered jet.

%Modeling of the emission from these objects has driven a very active field with models falling into both of the above categories (binary pulsar vs microquasar), as well as utilizing both leptonic and hadronic emission scenarios. As more ``direct'' attempts to measure the nature of the systems in question have yet to yield decisive information (for example, constraining the inclination angle of viewing to narrow the compact mass down to rule out a black hole) the way forward appears to be further monitoring of the systems across the spectrum in the hopes of discovering some observational feature that would firmly identify these systems and the basic interacting parts within them. 

The orbital periods of these objects vary from several days (LS 5039) to several years (PSR B129-63), and as a result, the various sources may only present short windows during which they can be studied in the TeV regime. Of the TeV binaries, \lsi{} is the only known source in the Northern Hemisphere which has a short enough orbital period (26.5 days) to allow for regular study with TeV instruments. This has made it an excellent target for Northern Hemisphere TeV observatories. 

\lsi{}, located at a distance of $\sim2$ kpc, is composed of a B0 Ve star and a compact object \citep{HandC1981, Casares2005}. The observed radio through TeV emission is variable and modulated with a period of $P \approx 26.5$ days, believed to be associated with the orbital structure of the binary system \citep{Albert2006, Esposito2007, VERITASLSIDetection, Abdo2009, LiXray, 2015A&A...575L...9M}. Radial velocity measurements show the orbit to be elliptical ($e = 0.537\pm0.034$), with periastron passage occurring around phase $\phi=0.275$, apastron passage at $\phi=0.775$, superior conjunction at $\phi=0.081$ and inferior conjunction at $\phi=0.313$ \citep{Aragona2009}. However, the inclination of the system is not exactly known, leading to some uncertainty of the orbital parameters.

As a TeV source, \lsi{} has presented puzzling behavior. Initial detections in 2006\,--\,2007 by both the MAGIC \citep{Albert2006} and VERITAS \citep{VERITASLSIDetection} collaborations over many orbital cycles showed the source to be a variably bright TeV source, with emission peaking around apastron passage. Subsequent observations in 2008\,--\,2010 \citep{2011ApJ...738....3A} showed no evidence for emission during these previously detected phases, instead only detecting the source at a lower TeV flux near the periastron passage of a single orbit. 

However, VERITAS observations taken in Nov\,--\,Dec 2011 showed the source to be highly active around apastron again \citep{2013ApJ...779...88A}, similar to the behavior observed in 2006\,--\,2007. Since 2011, observations of \lsi{} by VERITAS have only revealed typical emission levels, i.e., 5\,--\,15\% of the Crab Nebula flux, with emission peaking around apastron. In this work we present the results of the VERITAS campaign on \lsi{} in the Fall of 2014. During this time, VERITAS observed historically bright flares from \lsi{} around apastron, with the source exhibiting flux levels a factor of 2\,--\,3 times higher than previously seen.

\section{Observations}
The VERITAS IACT array, located at the base of Mt. Hopkins, AZ (1.3 km a.s.l., 31$^{\circ}$40'\,N, 110$^{\circ}$57'\,W) consists of four 12\,m diameter Davies-Cotton design optical telescopes. VERITAS is sensitive from 85\gev{} to 30\tev{}, and has the ability to detect a 1\,\% Crab Nebula source in approximately 25 hours\footnote{\url{http://veritas.sao.arizona.edu/about-veritas-mainmenu-81/veritas-specifications-mainmenu-111}}. For a full description of the hardware components and analysis methods utilized by VERITAS, see \citet{VERITAS, KiedaVTSUpgrade, VERITASLSIDetection}, and references therein.

In the 2014 season, VERITAS observations of \lsi{} were taken from October 16 (MJD 56946) to  December 12 (MJD 57003), obtaining a total of 24.7 hours of quality selected livetime. These observations covered three separate orbital periods of \lsi{}, sampling the orbital regions of $\phi = 0.5-0.2$ (see Figure~\ref{f:fig1}). Over the entire set of observations, a total of 449 excess events above background were detected, equivalent to a significance of $21\sigma$ calculated using Equation 17 of \citet{LiMa}.

\begin{figure}[ht]
\centering
%\subfigure[\label{fig1}]{
%  \includegraphics[width=0.5\textwidth]{./figs/Figure1-eps-converted-to.pdf}
%  %\caption{The VERITAS $>$350 GeV light curve of \lsi{} during the 2014 observation season (\textbf{VEGAS}).} 
%  %\label{fig1}
%}
%\subfigure[\label{fig1b}]{
%  \includegraphics[width=0.5\textwidth]{figs/LSI-mod-lc-days-gt350gev.png}
%  %\caption{The VERITAS $>$350 GeV light curve of \lsi{} during the 2014 observation season (\textbf{ED}).}
%  %\label{fig1b}
%}
\includegraphics[width=0.5\textwidth]{./figs/fluxvphase.pdf}
\caption{Light curve of \lsi{} during the 2014 observation season %from \textbf{VEGAS}~\subref{fig1} and \textbf{ED}~\subref{fig1b}. 
\textbf{Describe axes and the UL criteria.}}
\label{f:fig1}
\end{figure}

During the first orbit observed (in October), the source presented the largest of its flares (hereafter ``F1''), beginning on 17 October 2014 (MJD 56947, $\phi = 0.55$) with emission reaching a peak of $3.0 \pm 0.3 \times10^{-11}$\pflux{} ($>$350 GeV) on October 18 (MJD 56948). This flare peaked at approximately $30\%$ of the Crab Nebula flux in the same energy range, representing the largest flux ever detected from the source. Unfortunately, observations were limited by poor weather conditions for two nights following this peak and only recommenced on October 20 (MJD 56950), by which time the source had already quietened down. As can be seen in Figure~\ref{f:fig1}, this flare is rather sharply defined: only one night before the flare began, the source was in a quescient state with a $99\%$ confidence upper limit of emission of $1.4 \times10^{-11}$\pflux{} ($>$350 GeV). \textbf{This short variability fast confirms the result of \citep{2013ApJ...779...88A}.}

During the November observation (second orbital passage), VERITAS detected another period of high activity for the source at similar orbital phases ($\phi = 0.5-0.6$) with similar flux levels detected. Follow-up observations conducted by VERITAS during the next month (2014 December 10\,--\,12) covered the orbital phases of $\phi=0.59-0.67$ and detected the source at a lower flux level of $\sim1.7\times10^{-11}$\pflux{} ($>$350 GeV).

During the 2014 observing season, the differential energy spectrum of \lsi{} was consistent with past observations, i.e., the emission in the $0.2-25$\tev{} range is well fit by a power-law described by $\left( 1.70 \pm 0.69_{\mathrm{stat}} \right) \times 10^{-12} \cdot \left( \frac{E}{\mathrm{1 TeV}} \right)^{-2.35 \pm 0.32_{\mathrm{stat}}}$ cm$^{-2}$ s$^{-1}$ TeV$^{-1}$, as can be seen in Figure~\ref{spec}.

\begin{figure}[ht]
\centering
\includegraphics[width=0.5\textwidth]{figs/vegasspec.pdf}
\caption{Average differential energy spectrum of \lsi{}.}
\label{spec}
\end{figure}

During these observations, the source was also monitored by \emph{Fermi}-LAT (0.1\,--\,300 GeV), \emph{Swift}-XRT (0.2\,--\,10 keV), and both the RATAN and AMI radio instruments (4/6\,--\,15 Ghz). In addition, H-alpha monitoring of the system took place with the Ritter Observatory in Toledo, Ohio (USA). After the second flare (November) was detected by VERITAS, Atel $\#$6785 \textbf{ref} was released, notifying the astronomical community of the historic flux levels and triggering more intense observations with the existing multiwavelength partners, as well as additional observations with the MAGIC TeV observatory. The results of this campaign are under analysis and will be presented in an upcoming publication. 

\section{Discussion and Interpretation}

Following number calculated from ED analysis - need numbers from VEGAS for the final version!
\begin{itemize}
  \item The VHE light curve is inconsistent with a constant flux at $14.6\,\sigma$
  \item Orbit 1 is inconsistent with a constant flux at $8.3,\sigma$
  \item Orbit 2 is inconsistent with a constant flux at $8.4\,\sigma$
%  \item When folded with the orbital period of 26.5 days found at other wavelengths, the VHE light curves shows modulation significant at ??$\sigma$
  \item No evidence for intra-night variability
    \begin{itemize}
      \item Peak night of F1 fit with constant flux model, $\chi^2/\mbox{\textit {NDF}} = 8.186/2$, $P = 0.01669$, $\rightarrow$ inconsistent with model at $2.4\,\sigma$
      \item Peak night of F2 fit with constant flux model, $\chi^2/\mbox{\textit {NDF}} = 7.539/4$, $P = 0.11$, $\rightarrow$ inconsistent with model at $1.6\,\sigma$
      \item $\Rightarrow$ accept model for both cases
    \end{itemize}
  \item Orbit 1 and 2 (in daily bins, as now we know there is not intra-night variability) fit with a constant + a Gaussian to accommodate a flare superimposed on a base line flux
  \begin{itemize}
    \item Orbit 1: Fixed mean of Gaussian to 56947.8 (highest measured flux point), as the fit did not converge otherwise. Obtained $\chi^2/\mbox{\textit {NDF}} = 16.97/8$, $P = 0.03046$, with a $\mbox{\textit{FWHM}} = 0.665 \pm 0.0987$ for a rise and fall time of $\sim2$ days.
    \item Orbit 2: All parameters remained free, mean of Gaussian naturally converged to highest measured flux point. Obtained $\chi^2/\mbox{\textit {NDF}} = 6.298/8$, $P = 0.6139$, with a $\mbox{\textit{FWHM}} = 0.853 \pm 0.112$ for a rise and fall time of $\sim2.5$ days.
  \end{itemize}
  \item Shortest significant flux variability timescale is 1.8 days, at $7.15\,\sigma$ (between 1st and 3rd night of F1)
  \begin{itemize}
    \item 1-day variability at $3.56\,\sigma$ (this is max sigma for 1-day var, measured between 1st and 2nd night of F1)
  \end{itemize}
  \item Check spectral index variability
  \item The highest energy photon is ~10.55 TeV
  \item Optical depth?
  \item Could short-term var be due to sudden changes in transperency?
\end{itemize}


\bibliography{refs}


\end{document}

%%
%% End of file `sample.tex'.
